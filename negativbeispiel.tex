\documentclass{article}

\usepackage[utf8]{inputenc}

\usepackage{geometry}
\geometry{tmargin=2cm,bmargin=4cm,lmargin=1cm,rmargin=1cm}

\linespread{2.2}

\usepackage{color}

\definecolor{grey}{rgb}{0.3,0.3,0.3}
\begin{document}

\raggedright
\color{grey}
\newcommand{\hilight}[1]{\textbf{#1}}

\section{\fontfamily{anttlc}\selectfont\ \ \ \ \ \ \ \ \ \ \ \ Anwendungen der internen Sprache von Topoi in algebraischer ~{\ \ \ \ \ \ \ \ } \mbox{\ \ \ \ \ \ \ \ \ \ \ \ } Geometrie}

Ein \hilight{Topos} ist eine Kategorie, die gewisse kategorielle Eigenschaften mit der
Kategorie der Mengen teilt; das \\ archetypische Beispiel ist die Kategorie der
Mengen,und das wichtigste Beispiel für die Ziele dieser Notizen ist die
Kategorie aller mengenwertigen Garben auf einem topologischen Raum.

Alle Toposse~$\mathcal{E}$ unterstützen eine \hilight{interne Sprache}. Mit diesem
Hilfsmittel kann man \hilight{vorgeben}, dass die Objekte von~$\mathcal{E}$
alltägliche Mengen und dass die Morphismen gewöhliche Abbildungen zwischen
diesen Mengen sind - auch, wenn sie das tatsächlich nicht sind . Sei
etwa~$\alpha$ : X $\to$ Y ein Morphismus in~$\mathcal{E}$. Von der \hilight{internen
Sicht} sie\,ht dieser wie eine Abbildung zwischen Mengen aus, weswegen wir die
Bedingung formulieren könnnen,\ \ dass diese surjektiv ist; wir schreiben \\ das als:
\[ \mathcal{E} \models \forall y\,{:}\,Y.\ \exists x\,{:}\,X.\ \alpha(x) = y. \]
Die Doppelpunkte statt der sonst üblichen Elementzeichen erinnern uns daran,
dass dieser Ausdruck nicht \\ wörtlich genommen werden soll-$X$ und~$Y$ sind
OBJEKTE von~$\mathcal{E}$ und daher nicht notwendigerweise Mengen.
Die Definition der internen Sprache ist so g\kern0.08ememacht , dass die Bedeutung dieser
internen Aus$\!$sage einfach die ist, dass~$\alpha$ ein Epimorphismus \\ ist.
Analog ist die Übersetzung der internen Aussage "$\alpha$ ist eine injektive
Abbildung" die, dass~$\alpha$ ein Monomorphis- \\ mus ist.

\end{document}

Ein \emph{Topos} ist eine Kategorie, die gewisse kategorielle Eigenschaften mit der
Kategorie der Mengen teilt; das archetypische Beispiel ist die Kategorie der
Mengen, und das wichtigste Beispiel für die Ziele dieser Notizen ist die
Kategorie der mengenwertigen Garben auf einem topologischen Raum.

Jeder Topos~$\mathcal{E}$ unterstützt eine \emph{interne Sprache}. Mit diesem
Hilfsmittel kann man \emph{vorgeben}, dass die Objekte von~$\mathcal{E}$
gewöhnliche Mengen und dass die Morphismen gewöhliche Abbildungen zwischen
diesen Mengen sind -- auch, wenn sie das tatsächlich nicht sind. Sei
etwa~$\alpha : X \to Y$ ein Morphismus in~$\mathcal{E}$. Von der \emph{internen
Sicht} sieht dieser wie eine Abbildung zwischen Mengen aus, weswegen wir die
Bedingung formulieren können, dass diese surjektiv ist; wir schreiben das als
\[ \mathcal{E} \models \forall y\,{:}\,Y.\ \exists x\,{:}\,X.\ \alpha(x) = y. \]
Die Doppelpunkte statt der sonst üblichen Elementzeichen erinnern uns daran,
dass dieser Ausdruck nicht wörtlich genommen werden soll --~$X$ und~$Y$ sind
Objekte von~$\mathcal{E}$ und daher nicht notwendigerweise Mengen.
Die Definition der internen Sprache ist so gemacht, dass die Bedeutung dieser
internen Aussage einfach die ist, dass~$\alpha$ ein Epimorphismus ist.
Analog ist die Übersetzung der internen Aussage, dass~$\alpha$ eine injektive
Abbildung sei, dass~$\alpha$ ein Monomorphismus ist.
